\chapter{RelatedWork}
\label{chap:RelatedWork}

\section{Software Naturalness}
The primary research in the field started with a presumption that software is natural.
The foreleading paper on the topic (Hindle, Abram, et al. "On the naturalness of software", 2012)[1],
explains it like this: "software is natural, in the sense that it is created by humans at work,
with all the attendant constraints and limitations — and thus, like natural language, it is also
likely to be repetitive and predictable". Not only that, further research revealed that code is
even more predictable and full of patterns than human languages. In the paper, they also go as far
as to prove that (a) code can be usefully modelled by statistical language models and (b) such models
can be leveraged to support software engineers.

Allamanis et al. in their paper "A survey of machine learning for big code and naturalness"[2]
say the following: "software engineering and programming languages (SE/PL) should make the same
transition [as natural language processing (NLP) research], augmenting traditional methods that
consider only the formal structure of programs with information about the statistical properties
of code", also exploiting recurring and predictable patterns in code.

First, such tries were made by Hindle et al. [1]. Their solution was capturing high-level statistical
regularity at the n-gram level (probabilistic chains of tokens), by taking n-1 previous tokens that
are already entered into the text buffer, and attempting to guess the next token. The model built
from the corpus gives maximum likelihood estimates of the probability of a specific choice of next token;
this probability can be used to rank order the likely next tokens.

A fair share of the research following the n-gram model approach has been built on top of that.
For instance, Tu et al. [4] also "noticed that code has a high degree of localness, where identifiers
(e.g. variable names) are repeated often within close distance" [2]. Thus, they applied a cache mechanism
that assigns higher probability to tokens that have been observed most recently.

After that, researches have turned to deep learning, more particularly using RNNs (deep recurrent
neural networks) to outperform n-grams. These models predict each token sequentially, but loosen
the fixed context-size assumption, instead representing the context using a distributed vector
representation [2]. Karpathy et al. [5] used character-level LSTMs, however it was mostly to explore
the accuracy of predictions LSTMs make of the code patterns in general.

P.S. this section is largely taken from my blog (\href{https://medium.com/@myroslavarm/machine-learning-for-code-completion-2583792997e3}{link}).

\section{Overview of other papers}
Hellendoorn, Vincent J., et al. in their paper "When code completion fails: A case study on real-world
completions" recorded the results of a case study of 15,000 completions (completion events) for VisualStudio.
The conclusions they've reached based on their analysis are as follows:
\begin{itemize}
	\item synthetic benchmarks misrepresent real-world completions
    \item models that dynamically integrate local data fare much better than static ones, but intraproject queries remain the hardest category
    \item benchmarks should address completions following their frequency of use, showing that intra-project API completions are surprisingly relevant
    \item real-world efficacy is increased by focusing on the least accurate predictions in synthetic data: these are far more common and time-consuming IRL
    \item even though RNNs often outperform n-gram models in typical natural language settings, n-gram models are sometimes a better choice for modeling source code. For example, the deep learner is better at core method invocations but loses on third-party library calls, whereas the n-gram model naturally outperforms it on internal API calls but loses out on the other categories
    \item API recommendation tools that only recommend public APIs fail to address over half of the real-world queries
\end{itemize}
Their approach for the case study was: a static vocabulary is estimated on the training data and
all events seen less than 10 times are treated as a generic "unknown" token, to produce a vocabulary
of 75,913. Conclustions: approaches to perform code completions span a wide spectrum of techniques,
but, virtually, all intelligent code completion models attempt to relate the context at the site of a
required completion to a context that has been observed in some large training corpus. And completions
can be split into two categories: Structural Feature Selection (domain knowledge of source code)
and Linguistic Models of Code (ML/statistical models).


Proksch, et al. in their paper "Intelligent Code Completion with Bayesian Networks" worked on
an extensible inference engine for intelligent code completion systems, called PatternBased 
Bayesian Network (PBN) -- Eclipse Code Recommenders6 project adapted the PBN approach for their
intelligent call completion. They've also tested (evaluating quality, speed and model size)
Best Matching Neighbour algorithm, saying "We introduce Bayesian networks as an alternative
underlying model, use additional context information for more precise recommendations, and
apply clustering techniques to improve model sizes.
They've reached the following conclusions:
\begin{itemize}
    \item showing the developer hundreds of recommendations (e.g., the type Text in the SWT framework of Eclipse1 lists 168 methods and field declarations) may be as ineffective as showing none
    \item intelligent code completions better target the needs of developers that are unfamiliar with an API (example: The FrUiT tool [Bruch et al. 2006])
    \item two further important quality dimensions for code completion engines to effectively support developers: inference speed and model size
    \item it is possible to see saturation effects starting at about 1,000 object usages, i.e. every tripling of the input size leads to a smaller increase in prediction quality
\end{itemize}
Check out: the system proposed by Heinemann et al. (2012) recommends methods to a developer that are relevant in the class currently under development.
This article showed that parameter call sites and the enclosing class context do not contribute
much to prediction quality. However, this information could be crucial to find the right proposals
for some types. Machine learning algorithms could detect and remove outliers in the dataset
to further improve prediction quality.

\section{More Papers}
"Are Deep Neural Networks the Best Choice for Modeling Source Code?"
We introduce a dynamically updatable, nested scope, unlimited vocabulary count-based
N-gram model that significantly outperforms all existing token-level models, including
very powerful ones based on deep learning. Our work illustrates that traditional
approaches, with some careful engineering, can beat deep learning models.

"Code Completion with Neural Attention and Pointer Networks"
In this paper, we apply neural language models on the code completion task, and
develop an attention mechanism which exploits the parent-children information on
program’s AST. To deal with the OoV values in code completion, we propose a pointer
mixture network which learns to either generate a new value through an RNN component,
or copy an OoV value from local context through a pointer component. Experimental
results demonstrate the effectiveness of our approaches.

"NEURAL CODE COMPLETION"
However, existing approaches to intelligent code completion either rely on strong typing
(e.g., Visual Studio for C++), which limits their applicability to widely used dynamically
typed languages (e.g., JavaScript and Python), or are based on simple heuristics and term
frequency statistics which are often brittle and are relatively error-prone.
Raychev et al. (2014 -- Code completion with statistical language models) and White et al.
(2015 -- Toward deep learning software repositories) explore how to use recurrent neural networks
(RNNs) to facilitate the code completion task. However, these works only consider running
RNNs on top of a token sequence to build a probabilistic model. Although the input sequence
considered in Raychev et al. (2014) is produced from an abstract object, the structural
information contained in the abstract syntax tree is not directly leveraged by the RNN structure
in both of these two works. In contrast, we consider extending LSTM, a RNN structure,
to leverage the structural information directly for the code prediction task.

"Code Completion with Statistical Language Models"
In this paper we presented a new approach to code completion based on a novel combination
of program analysis with statistical language models, and implemented that approach in a tool
called SLANG. Given a massive codebase, using program analysis, SLANG first extracts abstract
histories from the data. Then, these histories are fed to a language model such as an n-gram
model or recurrent neural network model, which treats the histories as sentences in a natural
language and learns probabilities for each sentence.

MYROSLAVA CONCLUSION: seems most of this research hasn't been actually implemented; these are artificial
studies to check how to improve ML use for code completion without any real implementation.
This needs to be checked for sure, but if this is indeed the case, we have an advantage with our
approach, because if we manage to implement it, we will be in the minority. But of course, the
research for IDEs (current usage) still needs to be explored.

"A Statistical Semantic Language Model for Source Code"
Eclipse [4] and IntelliJ IDEA [12, 11] support template-based completion for common
constructs/APIs (for/while, Iterator).
https://javascript.software.informer.com/download-javascript-code-completion-tool-for-eclipse-plugin/
and intellisense

"Learning from Examples to Improve Code Completion Systems"
Actually integrated their implementation (BMN based) into Eclipse, but it's 2009.

to read : "Products, Developers, and Milestones: How Should I Build My N-Gram Language Model"


\section{References}
\begin{enumerate}
	\item Hindle, Abram, et al. "On the naturalness of software." 2012 34th International Conference on Software Engineering (ICSE). IEEE, 2012. \href{http://citeseerx.ist.psu.edu/viewdoc/download?doi=10.1.1.221.1261&rep=rep1&type=pdf}{link}
	\item Allamanis, Miltiadis, et al. "A survey of machine learning for big code and naturalness." ACM Computing Surveys (CSUR)51.4 (2018): 81. \href{https://arxiv.org/pdf/1709.06182.pdf}{link}
	\item Amann, Sven, et al. "A study of visual studio usage in practice." 2016 IEEE 23rd International Conference on Software Analysis, Evolution, and Reengineering (SANER). Vol. 1. IEEE, 2016. \href{https://sarahnadi.org/resources/pubs/Amann_SANER16.pdf}{link}
	\item Zhaopeng Tu, Zhendong Su, and Premkumar Devanbu. "On the localness of software." Proceedings of the 22nd ACM SIGSOFT International Symposium on Foundations of Software Engineering. ACM, 2014. \href{http://zptu.net/papers/fse2014_localness.pdf}{link}
	\item Karpathy, Andrej, Justin Johnson, and Li Fei-Fei. "Visualizing and understanding recurrent networks." arXiv preprint arXiv:1506.02078 (2015). \href{https://arxiv.org/pdf/1506.02078.pdf}{link}
	\item Proksch, Sebastian, Johannes Lerch, and Mira Mezini. "Intelligent code completion with Bayesian networks." ACM Transactions on Software Engineering and Methodology (TOSEM) 25.1 (2015): 1-31. (\href{https://ieeexplore.ieee.org/stamp/stamp.jsp?arnumber=8812116}{link})
	\item Hellendoorn, Vincent J., et al. "When code completion fails: A case study on real-world completions." 2019 IEEE/ACM 41st International Conference on Software Engineering (ICSE). IEEE, 2019. (\href{https://dl.acm.org/doi/pdf/10.1145/2744200}{link})
\end{enumerate}