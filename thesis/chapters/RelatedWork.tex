\chapter{RelatedWork}
\label{chap:RelatedWork}

\section{Software naturalness}
In the famous paper "On The Naturalness of Software", \cite{Hind12a} compared source code
to natural languages. They claim that code is even more repetitive, predictable and full of
patterns than human languages. In the paper, they also argue that code can be modelled by
statistical language models, which can be used to support software engineers. Their approach
was based on capturing high-level statistical regularity at the n-gram level by taking n-1
previous tokens that are already entered into the text buffer, and attempting to guess the
next token. Using this model, it is possible to estimate the most probable sequences of tokens
and suggest the most relevant code completions to developers.

The work by \cite{Hind12a} served as a sort of catalyst for the following research. For
instance, \cite{Tu14a} also learnt that code "has a high degree of localness, where identifiers
(e.g. variable names) are repeated often within close distance", from \cite{Alla18a}. Thus,
they applied a cache mechanism that assigns higher probability to tokens that have been observed
most recently. In the more recent years, researches started applying deep learning models such
as deep recurrent neural networks. These models predict each token sequentially, but loosen the
fixed context-size assumption, instead representing the context using a distributed vector
representation, from \cite{Alla18a}.

\section{Conclusions from further experiments}
\cite{Hell19a} recorded the results of a case study of 15,000 completions
(completion events) for VisualStudio. One of the conclustions they've reached is that even though
RNNs often outperform n-gram models in typical natural language settings, n-gram models are
sometimes a better choice for modeling source code. For example, the deep learner is better at
core method invocations but loses on third-party library calls, whereas the n-gram model
naturally outperforms it on internal API calls but loses out on the other categories.

\cite{Prok15a} worked on an extensible inference engine for intelligent code completion
systems, called PatternBased Bayesian Network (PBN). Eclipse Code Recommenders6 project adapted
the PBN approach for their intelligent call completion. They've also tested (evaluating quality,
speed and model size) Best Matching Neighbour algorithm, using additional context information
for more precise recommendations, and applying clustering techniques to improve model sizes.
They've reached the following conclusions: showing the developer hundreds of recommendations
may be as ineffective as showing none; intelligent code completions better target the needs of
developers that are unfamiliar with an API; at about 1,000 object usages, i.e. every tripling
of the input size, they get a smaller increase in prediction quality.

\cite{Hell17a} introduced a dynamically updatable, nested scope, unlimited vocabulary
count-based N-gram model which outperformed both the traditional N-gram models and the deep
learning RNN and LSTM models.

\cite{Li17a} developed an attention mechanism which exploits the parent-children information of
the code AST. As correctly predicting out-of-vocabulary (OoV) values in code completion is largely
unsuccessful, the authors implemented a pointer mixture network which either generates a new value
through an RNN component, or copies an OoV value from local context through a pointer component.

\cite{Rayc14a} implemented a tool called SLANG, which first extracts abstract histories from the data.
Then, these histories are fed to a language model such as an n-gram model or recurrent neural network
model, which treats the histories as sentences in a natural language and learns probabilities for
each sentence, not at all taking into considering code AST information.

\section{More Papers}
"Neural Code Completion"
Existing approaches to intelligent code completion either rely on strong typing (e.g.,
Visual Studio for C++), which limits their applicability to widely used dynamically typed
languages, or are based on simple heuristics and term frequency statistics which are often
brittle and are relatively error-prone. Raychev et al. (2014, Code completion with statistical
language models) and White et al. (2015, Toward deep learning software repositories) explore
how to use recurrent neural networks (RNNs) to facilitate the code completion task. However,
these works only consider running RNNs on top of a token sequence to build a probabilistic
model. Although the input sequence considered in Raychev et al. (2014) is produced from an
abstract object, the structural information contained in the abstract syntax tree is not
directly leveraged by the RNN structure in both of these two works. In contrast, we consider
extending LSTM to leverage the structural information directly for the code prediction task.

MYROSLAVA CONCLUSION: seems most of this research hasn't been actually implemented; these are
artificial studies to check how to improve ML use for code completion without any real
implementation. This needs to be checked for sure, but if this is indeed the case, we have an
advantage with our approach, because if we manage to implement it, we will be in the minority.
But of course, the research for IDEs (current usage) still needs to be explored.

"A Statistical Semantic Language Model for Source Code"
Eclipse [4] and IntelliJ IDEA [12, 11] support template-based completion for common
constructs/APIs (for/while, Iterator).
https://javascript.software.informer.com/download-javascript-code-completion-tool-for-eclipse-plugin/
and intellisense

"Learning from Examples to Improve Code Completion Systems"
Actually integrated their implementation (BMN based) into Eclipse, but it's 2009.