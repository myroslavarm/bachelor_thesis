\chapter{Introduction}
\label{chap:Introduction}

\section{Context}
Code completion, also often referred to as autocompletion, is one of the most used features in any IDE. Whether it is being used for improving speed and accuracy of typing, or is used as an API guide, it is an essential part of any successful IDE and a fair share of code editors. It is one of the first things a developer notices as soon as they start coding, and it is something that can certainly "make or break" the workflow. Therefore, it is no wonder that it is also a somewhat popular research topic, where software engineers try to figure out a way to make code completion as effective, as accurate, and as fast as possible.

There are many approaches to handling the process of completing code. The classic strategies have been to use lexical models (completing based on the prefix the user is typing) or to use semantic models (use code structure analysis information to tailor completions). In the latest years statistical and machine learning (ML) models have also gained popularity. For the most part when using ML, source code analysis is also often disregarded, and the results of simply training the model on the code base and inferring all the possible code dependencies are used instead.

\section{Problem}
\label{sec:Introduction-Problem}
Specifically, in the case of Pharo (the same name refers to a dynamically-typed programming language based on Smalltalk, as well as an IDE for it; throughout the work it should be clear from context which meaning is used) in the last year and a half a lot of work has been carried out to improve code completion. However, there are more potential improvements from which developers can only benefit, such as improving the relevance of suggestions and hence reducing time and effort the developer spends to go through the list of completions.

The current implementation of code completion in Pharo is based on parsing and analysing abstract syntax tree (AST) information of source code, where we are able to determine the structure of the code and infer the type information where possible. Once we have a list of semantically suitable completion candidates, they are sorted alphabetically. However, this approach ignores the potential of suggesting completion results based on code history, which is something that has been successfully adopted by other IDEs.

\section{Our Approch}
\label{sec:Introduction-Approach}
 In this work we intend to study how code completion in Pharo can be improved using statistical language models. In particular, the idea is to use the N-gram model, as it has been documented to be used for such a task (\cite{Hind12a}), and, if successfully adapted to Pharo, might help enhance the quality of code completion.

 In short, in the course of this work we intend to answer the following questions:
 \begin{itemize}
     \item Can we improve code completion in Pharo by sorting candidate completions with an n-gram language model?
     \item Can we build a tool based on a trained n-gram model that would propose completion fast enough to be used in an IDE?
     \item How can we numerically evaluate the results of code completion produced by different completion strategies?
 \end{itemize}
 