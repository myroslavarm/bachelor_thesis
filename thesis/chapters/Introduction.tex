\chapter{Introduction}
\label{chap:Introduction}

\section{Context}
\label{sec:Introduction-Context}
Code completion is one of the most used features in any IDE. Whether it is being used for improving speed and accuracy of typing, or is used as an API guide, helping developers find their way around libraries, code completion is one of the first things a developer notices as soon as they start coding. The speed with which the results are suggested, as well as their accuracy is very important, and it is something that can certainly "make or break" the workflow of the developer. Therefore, researchers and software engineers are always trying to find new approaches that can improve code completion quality and make it as effective, as accurate, and as fast\footnote{By "fast" we mean that the completion is quick to give suggestions, by "accurate" that the completion proposes contextually correct results, and by "effective" we mean that the combination of both the aforementioned qualities makes the tool successful in achieving the result, which is to find the most suitable completion as easily as possible.} as possible.

The main distinction among the many approaches to code completion would be whether the approach uses semantic context or not. Namely, semantic models are models that use code structure analysis to tailor completions to type and/or semantic role, or, in other words, to give contextually correct and relevant results. Machine learning models can also \remove{often} employ semantic analysis but \remove{very} often \insertion{they} treat source code \replace{more like simple}{as plain} text, with context not taken into consideration.

Throughout this project, I focused on the code completion in Pharo\footnote{\url{http://pharo.org/}}. Pharo is a dynamically typed programming language inspired by Smalltalk, and the Pharo IDE is an interactive IDE intended for developing in Pharo\footnote{\insertion{In the rest of this thesis, I will use the word "Pharo" to refer to the programming language, and "Pharo IDE" to refer to the development environment.}}.

The current code completion approach in the Pharo IDE is a semantic one: it is based on analysing the abstract syntax tree (AST) of source code. The parser finds the best suitable node for each part of code, and the completion engine suggests contextually relevant completions\insertion{:} \remove{for each one (i.e.} class names for a global node, method calls for a method node, and so on\remove{)}. The AST-based approach replaced the completion engine that had been \insertion{part of } the \remove{main Pharo} IDE \remove{implementation} since \replace{the start}{since Pharo was first released in 2008} \oz{I think it's a good chance to mention that Pharo is a young language that was released 12 years ago :)}; it used a less sophisticated parsing implementation and tried to infer the type by looking at kinds of messages sent to variables, and so on. Recently, a separate code completion engine that uses generators\footnote{Its main idea is to propose code completion suggestions from pre-generated local contexts using heuristics for optimisation. More details can be found at: \url{https://github.com/guillep/complishon}} was ported to Pharo as well.

\section{Problem}
\label{sec:Introduction-Problem}
As mentioned above, the current implementation of code completion in the Pharo IDE is based on the AST, which allows us to determine the structure of the code, get the semantic role of tokens, and infer their type where possible. As a result, only those completions that are correct in the given context are suggested.

However, currently there is no way to effectively sort the completion candidates. The order of suggestions is significant because with a wrong sorting strategy desired completions can appear at the bottom of the list\remove{, and with a regular completion window UI this means out of view of the developer}. This is unproductive as it will require the developer to scroll down or type more symbols. 

In this work the main goal is to come up with an effective sorting strategy that would show relevant results first. \remove{Thus,} \oz{Starting from here, it's no longer the Problem, but a description of your Proposed Approach. Move it to the next section (and yes, it will be hard because in the next section you already say the same thing --- but that's why you should not say the same thing twice!!)} I propose to sort the completion candidates using n-gram models, as I believe this strategy has potential to improve the relevance of sorting in the Pharo IDE. Particularly, I implement the unigram and bigram sorting strategies. \remove{In the end,} I evaluate the effectiveness of this approach by comparing alphabetic, unigram, and bigram sorting (\remove{in this case,} with the alphabetic sorting acting as a random baseline).

\section{Proposed Approach in a Nutshell}
\label{sec:Introduction-Approach}
In this work I study how code completion can be improved using statistical language models for sorting the results, on top of an existing semantic completion. In particular, I use the n-gram language model, as it has been documented to be used for such a task (\cite{Hind12a}), to find the most relevant completion suggestions based on their probability of occurring in source code. I \remove{then} implement two sorting strategies based on \remove{the results of training the} unigram and bigram language models \insertion{trained on a large corpus of source code collected from 50 projects written in Pharo (\cite{Zait20a})} \remove{on Pharo source code}. I \remove{then} evaluate whether this approach does indeed enhance the quality of code completion in the Pharo IDE. \oz{This is part of the Problem, not the Proposed Approach. But you can not move it up because you only introduce n-gram models in this section. So I suggest that you add another subsection here titled Research Questions} As part of this work, the following research questions are considered:
\begin{RQ}
    \item Can we improve the accuracy of code completion in the Pharo IDE by sorting candidate completions with n-gram language models?
    \item How can we effectively evaluate the results of code completion enhanced by different sorting strategies?
\end{RQ}

\section{Contributions}
\label{sec:Introduction-Contributions}
Throughout the course of this project I made the following contributions:
\begin{itemize}
    \item Trained several n-gram language models on the tokenised source code of the Pharo programming language.
    \item Used those trained models to build a sorting tool that enhances code completion in the Pharo IDE by sorting the proposed suggestions according to their relevance in the given context.
    \item Made one of the two implemented sorters both effective and fast enough to actually be suitable for developer use (both implementations can be loaded from the \href{https://github.com/myroslavarm/CompletionSorting}{CompletionSorting} GitHub repository and used in the Pharo IDE).
    \item Proposed a combination of two approaches for evaluating the results of code completion: (1) a quantitative evaluation technique inspired by \cite{Robb08a} and (2) a qualitative evaluation where a case study of several completion scenarios is performed and analysed.
    \item Enhanced the \href{https://github.com/pharo-ai/NgramModel}{NgramModel} library, which is an open source library that implements n-gram language models in Pharo (4 accepted pull requests). \oz{It would make more sense if this contribution was first. Because "first I implemented the n-gram library" and "then I used it to train n-gram models"}
    \item Part of this work has been accepted for the Student Research Competition at <Programming> 2020. The conference was supposed to take place in Porto, Portugal in March and was cancelled due to COVID-19. Despite that, the abstract is still up for publication in the ACM Digital Library.
    \item \remove{Finally,} to my knowledge\insertion{,} this is the first implementation of a code completion engine in Smalltalk family of IDEs that uses machine learning for sorting completion candidates.
\end{itemize}

\section{Structure of the Thesis}
\label{sec:Introduction-Structure}
{\hypersetup{linkcolor=black}
\begin{description}
	\item [Chapter \ref{chap:RelatedWorks}. \nameref{chap:RelatedWorks}] \hfill \\
	Here \repetition{I give a detailed overview of} the research done around code completion in the last decade and a half. It includes both the solely software engineering approaches predating the use of machine learning for this task, as well as exclusively machine learning experiments of the recent years.
    \item [Chapter \ref{chap:PharoCompletion}. \nameref{chap:PharoCompletion}] \hfill \\
	In this chapter, \repetition{I give a detailed overview of} the implementation details behind the current completion engine in the Pharo IDE, challenges of code completion for a dynamically typed languages, as well as describe the idea behind the sorter plugin.
	\item [Chapter \ref{chap:NgramBackground}. \nameref{chap:NgramBackground}] \hfill \\
    Here I provide a theoretical background needed to understand this work, which includes the introduction to n-gram language models and challenges of their implementation.
    \item [Chapter \ref{chap:ProposedSolution}. \nameref{chap:ProposedSolution}] \hfill \\
	This chapter contains a detailed description of the final solution: data preparation, approaches taken to implement the n-gram sorters, and various engineering details.
	\item [Chapter \ref{chap:Evaluation}. \nameref{chap:Evaluation}] \hfill \\
    In this chapter, I go into depth about evaluation techniques and challenges, as well as detail the evaluation of the approaches taken.
    \item [Chapter \ref{chap:Conclusion}. \nameref{chap:Conclusion}] \hfill \\
	This chapter contains a summary of the results of this work and the shape it can take in the future.
\end{description}
}