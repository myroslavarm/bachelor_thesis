\chapter{Conclusion}
\label{chap:Conclusion}
\section{Discoveries}
\label{sec:Conclusion-Discoveries}
In this work, I proposed a machine learning-based technique for improving the completion engine in the Pharo IDE. The exact implementation used the n-gram language models to sort the completion candidates that are suggested to the user \replace{during coding}{as he or she types}. \remove{Specifically,} \insertion{I implemented} two sorting strategies \remove{were implemented}: \remove{the first} one \remove{was} based on the unigram model\remove{,} and \replace{the second}{another} one on the bigram model. \oz{The rest of this paragraph hardly fits into the Conclusion chapter. But it would be really nice to say this in the chapter where you describe your proposed solution} N-gram models of a higher order, such as 3- and 4-gram models, were not considered due \replace{the difficulty of implementation}{their computational complexity}: the \replace{size of the data}{frequency table} increases \replace{dramatically}{exponentially} with every new order of \textit{n}\replace{, and making it fast enough to be usable would have been a really time-consuming task beyond the scope of this project}{which makes the calculation of probabilities too slow for the task, where results must be updated at every keystroke}.

Concluding this work, the answers to the research questions originally stated in Section \ref{sec:Introduction-Approach} go as follows:
\begin{RQ}
    \item \textbf{Can we improve the accuracy of code completion in the Pharo IDE by sorting candidate completions with n-gram language models?} As a result of the evaluation performed, the unigram based sorter has been shown to have a significantly better result than the default sorter in the Pharo IDE. The implementation is also fast enough for comfortable developer usage, and is available by loading the \href{https://github.com/myroslavarm/CompletionSorting}{CompletionSorting} library. \oz{Never hide URL behind the link text. What if this document is printed on a paper? You can write "CompletionSorting library available at https://..."}
    \item \textbf{How can we effectively evaluate the results of code completion enhanced by different sorting strategies?} Even \wrong{with an evident resources limitation} \oz{???!!!}, it \wrong{has been possible to come up with an approach which adequately measures the performance of the sorters}. \oz{how do you define "adequately"? For example, I don't agree with you: I doubt that evaluation was adequate and think that it can be improved. So don't write this sentence} In particular, for the quantitative evaluation, inspired by \cite{Robb08a}, I \remove{was able to} simulate\insertion{d} the completion process as it would happen naturally, by generating source code sequence at various stages of typing, and comparing the results proposed to the ones in the codebase. The qualitative approach \replace{is of interest from the standpoint of being able}{allowed me} to compare the suggestions each sorting strategy proposes first-hand \oz{and...? This sentence sounds like it's not finished}.
\end{RQ}

\section{Directions of Future Work}
\label{sec:Conclusion-FutureWork}
\subsection{Enhancing the Bigram Sorter}

\subsection{Conducting a More Extensive Evaluation}