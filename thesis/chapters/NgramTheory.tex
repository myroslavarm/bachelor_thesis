\chapter{N-gram Background}
\label{chap:NgramBackground}

In this chapter, we go over the theoretical background for the n-gram models and the challenges in their implementation.

\section{N-gram Language Models}
\label{sec:NgramBackground-LanguageModels}
Predicting the next word in a sequence is a central problem for many areas of Natural Language Processing (NLP), including speech recognition, spelling correction, spam filtering, machine translation, and others.

Models that assign probabilities to sentences or sequences of words, and can be used to find the most likely continuation of a sequence, are called language models (\cite{Jura09a}). Among them, is the n-gram language model, which assigns probabilities to sequences out of \textit{n} words, called the n-grams. A one-word sequence is a unigram, pairs of words are referred to as 2-grams or bigrams, 3-grams or trigrams are sequences of three words, and so on. Examples of bigrams might be "he ate", "ate the", "the whole" and "whole pizza", whereas trigrams would look like "he ate the", "ate the whole", and "the whole pizza".

To describe the conditional probability of a word \textit{w} based on history \textit{h}, we use the following notation:
\begin{equation}
    P(w|h)
\end{equation}

Here is a more concrete example: if we have a sentence "he ate the whole pizza", and want to compute the probability of the word "pizza" given that the previous words are "he ate the whole", we can express it as a conditional probability:
\begin{equation}
    P(\text{pizza}|\text{he ate the whole})
\end{equation}

To estimate this probability, we can use relative frequency: we need to compute the number of occurrences of "he ate the whole", as well the number of occurrences of the sentence "he ate the whole pizza", and divide the latter by the former:
\begin{equation}
    P(\text{pizza}|\text{he ate the whole})=\frac{C(\text{he ate the whole pizza})}{C(\text{he ate the whole})}
\end{equation}

To compute the probability of a whole sequence, not just one word, we can use the chain rule of probability:
\begin{equation}
    P(w_1w_2...w_n) = P(w_1)P(w_2|w_1)P(w_3|w_1w_2)...P(w_n|w_1w_2w_3...w_{n-1})\label{eq:1}
\end{equation}
which means that the probability of the first word occurring is just its own probability, of the second word, is its probability of occurring given that the first word occurred, et cetera, and the joint probability of the whole sequence is the product of each words' probability.

It is worth noting that \textit{n} can be quite large, and so in a sizeable data corpora it can be a challenge to estimate the probabilities of large sequences. However, there is an assumption that to compute the probability of a word inside the sequence, the whole history is not needed, and we can approximate the probability only relying on the few history words (\textit{n-1}) closest to the word whose probability we are estimating. This is called a Markov assumption, and it is believed to hold true for n-grams. In other words, if we consider a bigram model, this means that we can reduce our conditional probabilities as follows:
\begin{equation}
    P(\text{pizza}|\text{he ate the whole}) \sim P(\text{pizza}|\text{whole})
\end{equation}
where, in the case of a bigram model, we only consider one previous word, in the case of a trigram, two previous words, and so on.

From this, the joint probability of a sequence (see Equation \ref{eq:1}) can be approximated the following way:
\begin{equation}
    P(w_1^n) \approx \prod_{x=1}^n P(w_x|w_{x-1})
\end{equation}

\section{Unknown Words and Smoothing}
\label{sec:NgramBackground-Smoothing}
A notable problem for using n-gram models is data sparsity. It can happen that the words that we encounter in the test data were not present in the training data, and so are unknown to our model. In that case, the probability of them occurring will be zero. In that case, the joint probability of the whole sequence containing such a word will also be zero, which can lead to the model discarding perfectly valid and commonly encountered sequences, and might hurt the model performance.

One way of dealing with this is to replace, for example, the bottom n\% of the training data with the unknown word token <UNK>, and then replace every unknown word in the test data with <UNK> as well, turning it into a known word with a probability above zero.

An alternative solution to the problem of unknown words is smoothing, which involves redistributing the general probabilities of all the words in the dataset, i.e. giving a bit of probability of more frequently encountered words to the unknown ones.

There are several smoothing algorithms, the simplest of which is Laplace smoothing. In this algorithm, we add one to all the counts (so all the counts are increased by one, and there are no zero counts), and then add $|V|$ (the size of vocabulary, i.e. number of unique words in it) to the denominator, to take into account the extra observations. So the smoothed unigram probability of a word \textit{w} looks the following:
\begin{equation}
    P(w)=\frac{c+1}{N+V}
\end{equation}
where \textit{c} is the count of the word \textit{w} and \textit{N} is the total number of words.

\section{Summary}
\label{sec:NgramBackground-Summary}
\begin{itemize}
    \item Language models assign probabilities to sentences and sequences of words, and can be used to predict the next word in a sequence.
    \item N-gram language models calculate conditional probabilities of words in a corpus given a limited history of \textit{n-1} previous words.
    \item The joint probability of a sequence is a product of each words' probability.
    \item N-gram models belong to the family of Markov models -- they make a restrictive assumption that every word in a sequence depends only on \textit{n-1} previous words, and any words beyond that window have little effect on the probability and can be ignored.
    \item Occasionally there are words that are not in the vocabulary, which means their probability of occurrence will be zero, and will bring the joint probability of a sequence to zero also and might hurt the performance of the model.
    \item To solve this problem, the process of smoothing is used. That way probability is redistributed among all the words, with a little bit taken off more frequent words and given to the unknown ones.
\end{itemize}