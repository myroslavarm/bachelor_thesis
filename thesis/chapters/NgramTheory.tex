\chapter{N-gram Background}
\label{chap:NgramBackground}

In this chapter\insertion{,} we go over the theoretical background for the N-grams models, the challenges in their \remove{successful} implementation\remove{, as well as how they can be evaluated}.

\section{N-gram Language Models}
\label{sec:NgramBackground-LanguageModels}
\insertion{Predicting next word in a sequence is a central problem for many areas of} Natural Language Processing (NLP) \remove{being able to predict upcoming words or assign probabilities to sequences of words, like phrases and sentences, is one of the essential things in at least several areas, such as}\insertion{, including} speech recognition, spelling correction, spam filtering, machine translation\insertion{,} and so on. \remove{Thus, }\notclear{by assigning probabilities to words or sequences of words you can improve your accuracy, by suggesting things that are more likely to appear in a given context.} \oz{I don't think this last sentence has any sense}

Models that assign \remove{such} probabilities to sentences or sequences of words\insertion{, and can be used to find the most likely continuation of a sequence,} are called language models (\cite{Jura09a}). Among them, is the n-gram language model, which assigns probabilities to sequences out of \textit{n} words, called the n-grams. \remove{Just} one word \replace{will be}{is} a unigram, \insertion{pairs of words} \remove{2-gram} are referred to as \insertion{2-grams or} bigrams, 3-grams \replace{are referred to as}{or} trigrams \insertion{are sequences of three words}, and so on. An example of \remove{a} bigram\insertion{s} might be "he ate", "ate the", "the whole" and "whole pizza", whereas \remove{a} trigram\insertion{s} would look like "he ate the", "ate the whole", and "the whole pizza". 

\remove{The idea of an }n-gram language model\insertion{s} \remove{could therefore be phrased as the following: to} estimate the likelihood of a word in a sequence, based on the "history", which is the previous \textit{(n-1)} words. \wrong{From this point, we will use the term n-gram to mean either the n-gram language model or the sequence of n words such model uses.} \oz{So you're saying that "from this point we will be inconsistent and use the same term to reference two different concepts". I strongly suggest that you use "n-gram" for a group of n words and "n-gram model" for a language model}

\section{N-grams}
\label{sec:NgramBackground-Ngrams}
To compute the \insertion{conditional} probability of a word \textit{w} based on history \textit{h}, we use the following formula:\oz{This is not a formula and it doesn't compute anything :)}
\begin{equation}
    P(w|h)
\end{equation}

\replace{To put it into}{Here is} a more concrete example, if we have a sentence "he ate the whole pizza", and want to compute the probability of the word "pizza" given that the previous words are "he ate the whole", we can \replace{simply do:}{express it as a conditional probability}
\begin{equation}
    P(\text{pizza}|\text{he ate the whole})
\end{equation}

\remove{In order }to estimate it, we need to compute the number of occur\insertion{r}ences of "he ate the whole", as well the number occur\insertion{r}ences of the sentence "he ate the whole pizza", and divide the latter by the former:
\begin{equation}
    P(\text{pizza}|\text{he ate the whole})=\frac{C(\text{he ate the whole pizza})}{C(\text{he ate the whole})}
\end{equation}

However, in large \remove{enough} data corpora, it can be a challenge to estimate such \notclear{large sequences} \oz{Where did a large sequence come from? You defined "history" as a sequence of n-1 words. So sequence can not be large unless n is large.You should make it clear that in this Section, "history" can include all previous words in a text and it is the specificity of n-gram models that they restrict history to n-1 previous words.}. \remove{Hence,} the idea behind the n-gram model, is that we \badstyle{don't} need a large history, but can approximate it using just the last few words (\cite{Jura09a}). This can be done \badstyle{as per the Markov assumption, which means} we can reduce our conditional probabilities to be approximately equal: \oz{What is "Markov assumption"? Why is it approximately equal? Are you sure? I don't believe you}
\begin{equation}
    P(\text{pizza}|\text{he ate the whole}) \sim P(\text{pizza}|\text{whole})
\end{equation}

\notclear{\remove{In general,} n-gram probabilities can be estimated by counting the occur\insertion{r}ences in the corpus and normalising them (as per the maximum likelihood estimate, or MLE for short).}\oz{This makes no sense}

\notclear{need help with adding MLE and relative frequenc\replace{e}{y} explanations?}}

\section{Evaluating Models}
\label{sec:NgramBackground-EvaluatingModels}
\remove{Now, evaluation is a really important step to take after developing and testing a model. One wants to be able to measure the performance and be able to say if it is an improvement on the current state, and by how much. Essentially, there are two ways of evaluation: extrinsic and intrinsic. Extrinsic means actually using the results of the model in an application and measuring how it works. The intrinsic evaluation, however, can measure the results without testing them out in the application, but simply by running the model on the so called test data. It is important to note, that the test data shouldn't be the same as the data the model was initially trained on, to not skew the accuracy measurements.}

\remove{As raw probability is not used for evaluation language models, when using the intrinsic evaluation something called perplexity is used. Perplexity is the inverse probability of a test set, normalised by the total number of words.}

\oz{This doesn't seem to fit into your thesis. Do you use perplexity metric? Evaluation has nothing to do with the theory of n-gram models. Did you take this from Hindle?}

\section{Smoothing}
\label{sec:NgramBackground-Smoothing}
\badstyle{There is\remove{, however,} the problem of incorrectly estimating probabilities of sequences, if they are not known to us} (i.e. \badstyle{weren't} in the training set). Those might be perfectly valid sequences that \notclear{will inadvertently have the probability of 0}. \oz{Why? (it is not clear because you forgot to explain that the joint probability of a sequence is calculated as a product of all conditional probabilities; this is why when one multiplier is 0, the joint probability - the product - will also be 0)} Not only can this hurt the performance of an application, but it can also lead \notclear{to division by 0}. \oz{How can it lead to the division by 0?}

So if the word is \remove{entirely} unknown, we can \replace{convert}{replace} it \replace{to}{with} the unknown word token <UNK> and count it as any other word in the dataset. However, if the word we are dealing with is generally known (does exist in the vocabulary) but appears in an unknown context, i.e. after \remove{or before} \oz{in this work, you don't do the lookahead - your history consists of previous words, not the next words} a word they never appeared \replce{before}{earlier}, we can use a procedure called smoothing, which means we redist\insertion{r}ibute the general probability \remove{a little bit}, to not have any sequences with probability equal to 0. \oz{This explanation is not clear}

There are several smoothing algorithms, the simplest of which is Laplace smoothing, which we will describe as an example. In this algorithm, \replace{one simply needs to}{we} add 1 to all the counts, before normalising them (so all the counts will be increased by one, and there will be no 0 counts, and no 0 probabilities). \oz{This is not correct. If we simply add 1 to the nominator, the probabilities will not sum up to 1. Laplace smoothing also adds $|V|$ (the size of a vocabulary = number of unique words in it) to denominator. Read this: \url{https://en.wikipedia.org/wiki/Additive_smoothing}} Laplace smoothing \badstyle{doesn't} \wrong{perform well enough to be used in practice}. \oz{Wrong! It is used a lot} A more useful algorithm based on a similar idea, would be add-k smoothing, which adds a fractional count k (for example, 0.1, 0.5, etc.) to the count, instead of 1. \oz{If I'm not mistaking, there is a restriction that k > 0}

\section{Summary}
\label{sec:NgramBackground-Summary}
\remove{In short:}

\oz{Capital letters}
\begin{itemize}
    \item language models assign probabilities to sentences and sequences of words, and \replace{thus predict more likely words}{can be used to predict next word in a sequence}
    \item n-gram language models \replace{assign probabilities to sequences of \textit{n} words}{calculate conditional probabilities of words in a corpus given a limited history of n-1 previous words}
    \item \replace{being a Markov model, the probability of an n-gram model can be approximated to depend just on the several last words, instead of the whole history}{n-gram models belong to the family of Markov models --- they make a restrictive assumption that every word in a sequence depends only on n-1 previous words, and any words beyond that window have little effect on probability and can be ignored}
    \item \remove{n-gram models can be evaluated either extrinsically, by being embedded and tested out in an application, or intrinsically, by testing them on test data and using perplexity}
    \item smoothing is needed in case words appear in an unknown context; that way probability is redistributed among all the sequences \notclear{to prevent skewing of data} \oz{How can it skew the data?}
\end{itemize}