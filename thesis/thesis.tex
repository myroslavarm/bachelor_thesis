%%%%%%%%%%%%%%%%%%%%%%%%%%%%%%%%%%%%%%%%%
% Bachelor/Master Thesis 
% LaTeX Template
% Version 2.5 (27/8/17)
%
% This template was downloaded from:
% http://www.LaTeXTemplates.com
%
% Version 2.x major modifications by:
% Vel (vel@latextemplates.com)
%
% This template is based on a template by:
% Steve Gunn (http://users.ecs.soton.ac.uk/srg/softwaretools/document/templates/)
% Sunil Patel (http://www.sunilpatel.co.uk/thesis-template/)
%
% Template license:
% CC BY-NC-SA 3.0 (http://creativecommons.org/licenses/by-nc-sa/3.0/)
%
%%%%%%%%%%%%%%%%%%%%%%%%%%%%%%%%%%%%%%%%%

%----------------------------------------------------------------------------------------
%	PACKAGES AND OTHER DOCUMENT CONFIGURATIONS
%----------------------------------------------------------------------------------------
% https://www.overleaf.com/9572798646vszsfmjxtyqc
\documentclass[
11pt, % The default document font size, options: 10pt, 11pt, 12pt
%oneside, % Two side (alternating margins) for binding by default, uncomment to switch to one side
english, % ngerman for German
singlespacing, % Single line spacing, alternatives: onehalfspacing or doublespacing
%draft, % Uncomment to enable draft mode (no pictures, no links, overfull hboxes indicated)
%nolistspacing, % If the document is onehalfspacing or doublespacing, uncomment this to set spacing in lists to single
%liststotoc, % Uncomment to add the list of figures/tables/etc to the table of contents
%toctotoc, % Uncomment to add the main table of contents to the table of contents
%parskip, % Uncomment to add space between paragraphs
%nohyperref, % Uncomment to not load the hyperref package
headsepline, % Uncomment to get a line under the header
%chapterinoneline, % Uncomment to place the chapter title next to the number on one line
%consistentlayout, % Uncomment to change the layout of the declaration, abstract and acknowledgements pages to match the default layout
oneside,]{BachelorMasterThesis} % The class file specifying the document structure

% \usepackage{graphicx} % for graphics
\usepackage{float} % for graphics if you want the figure in a particular place (add [H])
\usepackage{subcaption} % for sub figures

\usepackage[utf8]{inputenc} % Required for inputting international characters
\usepackage[T1]{fontenc} % Output font encoding for international characters

\usepackage{mathpazo} % Use the Palatino font by default

\usepackage[backend=bibtex,style=authoryear,natbib=true]{biblatex} % Use the bibtex backend with the authoryear citation style (which resembles APA)

\addbibresource{rmod,others} % The filename of the bibliography

\usepackage[autostyle=true]{csquotes} % Required to generate language-dependent quotes in the bibliography

\usepackage{xcolor} %%% for comments
\usepackage{ifthen} %%% for comments
\usepackage{amssymb} %%% for comments

% comments \nb{label}{color}{text}
\newboolean{showcomments}
\setboolean{showcomments}{true}
\ifthenelse{\boolean{showcomments}}
    {\newcommand{\nb}[3]{
        {\colorbox{#2}{\bfseries\sffamily\scriptsize\textcolor{white}{#1}}}
        {\textcolor{#2}{\sf\small$\blacktriangleright$\textit{#3}$\blacktriangleleft$}}}
     \newcommand{\version}{\emph{\scriptsize$-$Id$-$}}}
    {\newcommand{\nb}[3]{}
     \newcommand{\version}{}}
\newcommand{\oz}[1]{\nb{Oleks}{red}{#1}}
\newcommand{\md}[1]{\nb{Marcus}{blue}{#1}}
\newcommand{\mr}[1]{\nb{Myroslava}{olive}{#1}}

%----------------------------------------------------------------------------------------
%	MARGIN SETTINGS
%----------------------------------------------------------------------------------------

\geometry{
	paper=a4paper, % Change to letterpaper for US letter
	inner=2.5cm, % Inner margin
	outer=3.8cm, % Outer margin
	bindingoffset=.5cm, % Binding offset
	top=1.5cm, % Top margin
	bottom=1.5cm, % Bottom margin
	%showframe, % Uncomment to show how the type block is set on the page
}

%----------------------------------------------------------------------------------------
%	THESIS INFORMATION
%----------------------------------------------------------------------------------------

\thesistitle{Improving Code Completion in Pharo\\Using N-gram Language Models} % Your thesis title, this is used in the title and abstract, print it elsewhere with \ttitle
\supervisor{Oleksandr \textsc{Zaitsev} \\ Marcus \textsc{Denker}} % Your supervisor's name, this is used in the title page, print it elsewhere with \supname
\examiner{} % Your examiner's name, this is not currently used anywhere in the template, print it elsewhere with \examname
\degree{Bachelor of Science} % Your degree name, this is used in the title page and abstract, print it elsewhere with \degreename
\author{Myroslava \textsc{Romaniuk}} % Your name, this is used in the title page and abstract, print it elsewhere with \authorname
\addresses{} % Your address, this is not currently used anywhere in the template, print it elsewhere with \addressname

\subject{Data Science} % Your subject area, this is not currently used anywhere in the template, print it elsewhere with \subjectname
\keywords{} % Keywords for your thesis, this is not currently used anywhere in the template, print it elsewhere with \keywordnames
\university{\href{http://www.ucu.edu.ua}{Ukrainian Catholic University}} % Your university's name and URL, this is used in the title page and abstract, print it elsewhere with \univname
\department{\href{http://department.university.com}{Faculty of Applied Sciences}} % Your department's name and URL, this is used in the title page and abstract, print it elsewhere with \deptname
\group{\href{http://researchgroup.university.com}{Department of Computer Sciences}} % Your research group's name and URL, this is used in the title page, print it elsewhere with \groupname
\faculty{\href{http://faculty.university.com}{}} % Your faculty's name and URL, this is used in the title page and abstract, print it elsewhere with \facname

\AtBeginDocument{
\hypersetup{pdftitle=\ttitle} % Set the PDF's title to your title
\hypersetup{pdfauthor=\authorname} % Set the PDF's author to your name
\hypersetup{pdfkeywords=\keywordnames} % Set the PDF's keywords to your keywords
}

\begin{document}

\frontmatter % Use roman page numbering style (i, ii, iii, iv...) for the pre-content pages

\pagestyle{plain} % Default to the plain heading style until the thesis style is called for the body content

%----------------------------------------------------------------------------------------
%	TITLE PAGE
%----------------------------------------------------------------------------------------

\begin{titlepage}
\begin{center}

\vspace*{.06\textheight}
{\scshape\LARGE \univname\par}\vspace{1.5cm} % University name
\textsc{\Large Bachelor Thesis}\\[0.5cm] % Thesis type

\HRule \\[0.4cm] % Horizontal line
{\huge \bfseries \ttitle\par}\vspace{0.4cm} % Thesis title
\HRule \\[1.5cm] % Horizontal line
 
\begin{minipage}[t]{0.4\textwidth}
\begin{flushleft} \large
\emph{Author:}\\
\authorname % Author name - remove the \href bracket to remove the link
\end{flushleft}
\end{minipage}
\begin{minipage}[t]{0.4\textwidth}
\begin{flushright} \large
\emph{Supervisors:} \\
\supname % Supervisor name - remove the \href bracket to remove the link  
\end{flushright}
\end{minipage}\\[3cm]
 
\vfill

\large \textit{The thesis submitted in fulfillment of the requirements\\ for the degree of \degreename}\\[0.3cm] % University requirement text
\textit{in the}\\[0.4cm]
\groupname\\\deptname\\[1cm] % Research group name and department name
 
\vfill
\includegraphics[height=5cm]{UCU-Apps.png} % University/department logo - uncomment to place it

\vfill
{\large Lviv 2020}\\[3cm] % Date
 
\vfill
\end{center}
\end{titlepage}

%----------------------------------------------------------------------------------------
%	DECLARATION PAGE
%----------------------------------------------------------------------------------------

\begin{declaration}
\addchaptertocentry{\authorshipname} % Add the declaration to the table of contents
\noindent I, \authorname, declare that this thesis titled, \enquote{\ttitle} and the work presented in it are my own. I confirm that:

\begin{itemize} 
\item This work was done wholly or mainly while in candidature for a research degree at this University.
\item Where any part of this thesis has previously been submitted for a degree or any other qualification at this University or any other institution, this has been clearly stated.
\item Where I have consulted the published work of others, this is always clearly attributed.
\item Where I have quoted from the work of others, the source is always given. With the exception of such quotations, this thesis is entirely my own work.
\item I have acknowledged all main sources of help.
\item Where the thesis is based on work done by myself jointly with others, I have made clear exactly what was done by others and what I have contributed myself.\\
\end{itemize}
 
\noindent Signed:\\
\rule[0.5em]{25em}{0.5pt} % This prints a line for the signature
 
\noindent Date:\\
\rule[0.5em]{25em}{0.5pt} % This prints a line to write the date
\end{declaration}

\cleardoublepage

%----------------------------------------------------------------------------------------
%	QUOTATION PAGE
%----------------------------------------------------------------------------------------

\vspace*{0.2\textheight}

\noindent\enquote{\itshape Thanks to my solid academic training, today I can write hundreds of words on virtually any topic without possessing a shred of information, which is how I got a good job in journalism.}\bigbreak

\hfill Dave Barry

%----------------------------------------------------------------------------------------
%	ABSTRACT PAGE
%----------------------------------------------------------------------------------------

\begin{abstract}
\addchaptertocentry{\abstractname} % Add the abstract to the table of contents
\oz{A dummy comment}
Code completion is one of the essential features of any IDE that greatly improves the developer experience and productivity. Good code completion can both speed up the development process, as well as aid the developer in API exploration. On the other hand, having a slow or inaccurate completion can be very cumbersome. Thus, it is important to find a way to make it as effective as possible. \md{Another dummy comment}

The current implementation of code completion in the Pharo IDE is based on abstract syntax tree (AST) information of source code. This allows us to know the semantic role and, where possible, even the type of tokens. Hence, at every point in the completion, contextually relevant results are suggested. \mr{And the last dummy comment}

However, the completion candidates that are proposed are sorted alphabetically. We believe, this approach ignores the potential of suggesting more relevant completion results based on their occurence in code history, which is something that has been successfully adopted by other IDEs.

Therefore, in this work we study how code completion can be improved using statistical language models for sorting completion results. In particular, for our experiment we chose the n-gram language model. With the help of the n-gram implementation, we implement a sorting strategy where the most relevant results (based on likelihood analysis of source code) are shown first. This allows us to propose more accurate suggestions even without knowing the exact type of the token, and makes finding a desired suggestion much faster.

In the end, we also quantitatively evaluate the various sorting strategies implemented to be able to gauge performance accuracy for each one. 
\end{abstract}

%----------------------------------------------------------------------------------------
%	ACKNOWLEDGEMENTS
%----------------------------------------------------------------------------------------

\begin{acknowledgements}
\addchaptertocentry{\acknowledgementname} % Add the acknowledgements to the table of contents
The acknowledgments and the people to thank go here, don't forget to include your project advisor\ldots
\end{acknowledgements}

%----------------------------------------------------------------------------------------
%	LIST OF CONTENTS/FIGURES/TABLES PAGES
%----------------------------------------------------------------------------------------

\tableofcontents % Prints the main table of contents

\listoffigures % Prints the list of figures

\listoftables % Prints the list of tables

%----------------------------------------------------------------------------------------
%	ABBREVIATIONS
%----------------------------------------------------------------------------------------

\begin{abbreviations}{ll} % Include a list of abbreviations (a table of two columns)

\textbf{API} & \textbf{A}pplication \textbf{P}rogramming \textbf{I}nterface\\
\textbf{AST} & \textbf{A}bstract \textbf{S}yntax \textbf{T}ree\\
\textbf{IDE} & \textbf{I}ntegrated \textbf{D}evelopment \textbf{E}nvironment\\
\textbf{LSTM} & \textbf{L}ong \textbf{S}hort-\textbf{T}erm \textbf{M}emory\\
\textbf{ML} & \textbf{M}achine \textbf{L}earning\\
\textbf{NLP} & \textbf{N}atural \textbf{L}anguage \textbf{P}rocessing\\
\textbf{RNN} & \textbf{R}ecurrent \textbf{N}eural \textbf{N}etwork\\
\textbf{RQ} & \textbf{R}esearch \textbf{Q}uestion\\

\end{abbreviations}

%----------------------------------------------------------------------------------------
%	PHYSICAL CONSTANTS/OTHER DEFINITIONS
%----------------------------------------------------------------------------------------

\begin{constants}{lr@{${}={}$}l} % The list of physical constants is a three column table

% The \SI{}{} command is provided by the siunitx package, see its documentation for instructions on how to use it

Speed of Light & $c_{0}$ & \SI{2.99792458e8}{\meter\per\second} (exact)\\
%Constant Name & $Symbol$ & $Constant Value$ with units\\

\end{constants}

%----------------------------------------------------------------------------------------
%	SYMBOLS
%----------------------------------------------------------------------------------------

\begin{symbols}{lll} % Include a list of Symbols (a three column table)

$a$ & distance & \si{\meter} \\
$P$ & power & \si{\watt} (\si{\joule\per\second}) \\
%Symbol & Name & Unit \\

\addlinespace % Gap to separate the Roman symbols from the Greek

$\omega$ & angular frequency & \si{\radian} \\

\end{symbols}

%----------------------------------------------------------------------------------------
%	DEDICATION
%----------------------------------------------------------------------------------------

\dedicatory{For/Dedicated to/To my\ldots} 

%----------------------------------------------------------------------------------------
%	THESIS CONTENT - CHAPTERS
%----------------------------------------------------------------------------------------

\mainmatter % Begin numeric (1,2,3...) page numbering

\pagestyle{thesis} % Return the page headers back to the "thesis" style



% Little hack to fix minitoc
%\adjustmtc
%\adjustmtc
%\adjustmtc

% Include the chapters of the thesis as separate files from the Chapters folder
% Uncomment the lines as you write the chapters

\chapter{Introduction}
\label{chap:Introduction}

\section{Context}

context of the work

\section{Problem}
\label{sec:Introduction-Problem}

research problems
\chapter{RelatedWork}
\label{chap:RelatedWork}

\section{Software Naturalness}
The primary research in the field started with a presumption that software is natural.
The foreleading paper on the topic (Hindle, Abram, et al. "On the naturalness of software", 2012)[1],
explains it like this: "software is natural, in the sense that it is created by humans at work,
with all the attendant constraints and limitations — and thus, like natural language, it is also
likely to be repetitive and predictable". Not only that, further research revealed that code is
even more predictable and full of patterns than human languages. In the paper, they also go as far
as to prove that (a) code can be usefully modelled by statistical language models and (b) such models
can be leveraged to support software engineers.

Allamanis et al. in their paper "A survey of machine learning for big code and naturalness"[2]
say the following: "software engineering and programming languages (SE/PL) should make the same
transition [as natural language processing (NLP) research], augmenting traditional methods that
consider only the formal structure of programs with information about the statistical properties
of code", also exploiting recurring and predictable patterns in code.

First, such tries were made by Hindle et al. [1]. Their solution was capturing high-level statistical
regularity at the n-gram level (probabilistic chains of tokens), by taking n-1 previous tokens that
are already entered into the text buffer, and attempting to guess the next token. The model built
from the corpus gives maximum likelihood estimates of the probability of a specific choice of next token;
this probability can be used to rank order the likely next tokens.

A fair share of the research following the n-gram model approach has been built on top of that.
For instance, Tu et al. [4] also "noticed that code has a high degree of localness, where identifiers
(e.g. variable names) are repeated often within close distance" [2]. Thus, they applied a cache mechanism
that assigns higher probability to tokens that have been observed most recently.

After that, researches have turned to deep learning, more particularly using RNNs (deep recurrent
neural networks) to outperform n-grams. These models predict each token sequentially, but loosen
the fixed context-size assumption, instead representing the context using a distributed vector
representation [2]. Karpathy et al. [5] used character-level LSTMs, however it was mostly to explore
the accuracy of predictions LSTMs make of the code patterns in general.

P.S. this section is largely taken from my blog (\href{https://medium.com/@myroslavarm/machine-learning-for-code-completion-2583792997e3}{link}).

\section{Overview of other papers}
Hellendoorn, Vincent J., et al. in their paper "When code completion fails: A case study on real-world
completions" recorded the results of a case study of 15,000 completions (completion events) for VisualStudio.
The conclusions they've reached based on their analysis are as follows:
\begin{itemize}
	\item synthetic benchmarks misrepresent real-world completions
    \item models that dynamically integrate local data fare much better than static ones, but intraproject queries remain the hardest category
    \item benchmarks should address completions following their frequency of use, showing that intra-project API completions are surprisingly relevant
    \item real-world efficacy is increased by focusing on the least accurate predictions in synthetic data: these are far more common and time-consuming IRL
    \item even though RNNs often outperform n-gram models in typical natural language settings, n-gram models are sometimes a better choice for modeling source code. For example, the deep learner is better at core method invocations but loses on third-party library calls, whereas the n-gram model naturally outperforms it on internal API calls but loses out on the other categories
    \item API recommendation tools that only recommend public APIs fail to address over half of the real-world queries
\end{itemize}
Their approach for the case study was: a static vocabulary is estimated on the training data and
all events seen less than 10 times are treated as a generic "unknown" token, to produce a vocabulary
of 75,913. Conclustions: approaches to perform code completions span a wide spectrum of techniques,
but, virtually, all intelligent code completion models attempt to relate the context at the site of a
required completion to a context that has been observed in some large training corpus. And completions
can be split into two categories: Structural Feature Selection (domain knowledge of source code)
and Linguistic Models of Code (ML/statistical models).


Proksch, et al. in their paper "Intelligent Code Completion with Bayesian Networks" worked on
an extensible inference engine for intelligent code completion systems, called PatternBased 
Bayesian Network (PBN) -- Eclipse Code Recommenders6 project adapted the PBN approach for their
intelligent call completion. They've also tested (evaluating quality, speed and model size)
Best Matching Neighbour algorithm, saying "We introduce Bayesian networks as an alternative
underlying model, use additional context information for more precise recommendations, and
apply clustering techniques to improve model sizes.
They've reached the following conclusions:
\begin{itemize}
    \item showing the developer hundreds of recommendations (e.g., the type Text in the SWT framework of Eclipse1 lists 168 methods and field declarations) may be as ineffective as showing none
    \item intelligent code completions better target the needs of developers that are unfamiliar with an API (example: The FrUiT tool [Bruch et al. 2006])
    \item two further important quality dimensions for code completion engines to effectively support developers: inference speed and model size
    \item it is possible to see saturation effects starting at about 1,000 object usages, i.e. every tripling of the input size leads to a smaller increase in prediction quality
\end{itemize}
Check out: the system proposed by Heinemann et al. (2012) recommends methods to a developer that are relevant in the class currently under development.
This article showed that parameter call sites and the enclosing class context do not contribute
much to prediction quality. However, this information could be crucial to find the right proposals
for some types. Machine learning algorithms could detect and remove outliers in the dataset
to further improve prediction quality.

\section{More Papers}
"Are Deep Neural Networks the Best Choice for Modeling Source Code?"
We introduce a dynamically updatable, nested scope, unlimited vocabulary count-based
N-gram model that significantly outperforms all existing token-level models, including
very powerful ones based on deep learning. Our work illustrates that traditional
approaches, with some careful engineering, can beat deep learning models.

"Code Completion with Neural Attention and Pointer Networks"
In this paper, we apply neural language models on the code completion task, and
develop an attention mechanism which exploits the parent-children information on
program’s AST. To deal with the OoV values in code completion, we propose a pointer
mixture network which learns to either generate a new value through an RNN component,
or copy an OoV value from local context through a pointer component. Experimental
results demonstrate the effectiveness of our approaches.

"NEURAL CODE COMPLETION"
However, existing approaches to intelligent code completion either rely on strong typing
(e.g., Visual Studio for C++), which limits their applicability to widely used dynamically
typed languages (e.g., JavaScript and Python), or are based on simple heuristics and term
frequency statistics which are often brittle and are relatively error-prone.
Raychev et al. (2014 -- Code completion with statistical language models) and White et al.
(2015 -- Toward deep learning software repositories) explore how to use recurrent neural networks
(RNNs) to facilitate the code completion task. However, these works only consider running
RNNs on top of a token sequence to build a probabilistic model. Although the input sequence
considered in Raychev et al. (2014) is produced from an abstract object, the structural
information contained in the abstract syntax tree is not directly leveraged by the RNN structure
in both of these two works. In contrast, we consider extending LSTM, a RNN structure,
to leverage the structural information directly for the code prediction task.

"Code Completion with Statistical Language Models"
In this paper we presented a new approach to code completion based on a novel combination
of program analysis with statistical language models, and implemented that approach in a tool
called SLANG. Given a massive codebase, using program analysis, SLANG first extracts abstract
histories from the data. Then, these histories are fed to a language model such as an n-gram
model or recurrent neural network model, which treats the histories as sentences in a natural
language and learns probabilities for each sentence.

MYROSLAVA CONCLUSION: seems most of this research hasn't been actually implemented; these are artificial
studies to check how to improve ML use for code completion without any real implementation.
This needs to be checked for sure, but if this is indeed the case, we have an advantage with our
approach, because if we manage to implement it, we will be in the minority. But of course, the
research for IDEs (current usage) still needs to be explored.

"A Statistical Semantic Language Model for Source Code"
Eclipse [4] and IntelliJ IDEA [12, 11] support template-based completion for common
constructs/APIs (for/while, Iterator).
https://javascript.software.informer.com/download-javascript-code-completion-tool-for-eclipse-plugin/
and intellisense

"Learning from Examples to Improve Code Completion Systems"
Actually integrated their implementation (BMN based) into Eclipse, but it's 2009.

to read : "Products, Developers, and Milestones: How Should I Build My N-Gram Language Model"


\section{References}
\begin{enumerate}
	\item Hindle, Abram, et al. "On the naturalness of software." 2012 34th International Conference on Software Engineering (ICSE). IEEE, 2012. \href{http://citeseerx.ist.psu.edu/viewdoc/download?doi=10.1.1.221.1261&rep=rep1&type=pdf}{link}
	\item Allamanis, Miltiadis, et al. "A survey of machine learning for big code and naturalness." ACM Computing Surveys (CSUR)51.4 (2018): 81. \href{https://arxiv.org/pdf/1709.06182.pdf}{link}
	\item Amann, Sven, et al. "A study of visual studio usage in practice." 2016 IEEE 23rd International Conference on Software Analysis, Evolution, and Reengineering (SANER). Vol. 1. IEEE, 2016. \href{https://sarahnadi.org/resources/pubs/Amann_SANER16.pdf}{link}
	\item Zhaopeng Tu, Zhendong Su, and Premkumar Devanbu. "On the localness of software." Proceedings of the 22nd ACM SIGSOFT International Symposium on Foundations of Software Engineering. ACM, 2014. \href{http://zptu.net/papers/fse2014_localness.pdf}{link}
	\item Karpathy, Andrej, Justin Johnson, and Li Fei-Fei. "Visualizing and understanding recurrent networks." arXiv preprint arXiv:1506.02078 (2015). \href{https://arxiv.org/pdf/1506.02078.pdf}{link}
	\item Proksch, Sebastian, Johannes Lerch, and Mira Mezini. "Intelligent code completion with Bayesian networks." ACM Transactions on Software Engineering and Methodology (TOSEM) 25.1 (2015): 1-31. (\href{https://ieeexplore.ieee.org/stamp/stamp.jsp?arnumber=8812116}{link})
	\item Hellendoorn, Vincent J., et al. "When code completion fails: A case study on real-world completions." 2019 IEEE/ACM 41st International Conference on Software Engineering (ICSE). IEEE, 2019. (\href{https://dl.acm.org/doi/pdf/10.1145/2744200}{link})
\end{enumerate}
\chapter{Completion in Pharo}
\label{chap:Completion in Pharo}

\begin{itemize}
    \item overview of the pharo ide and how completion works in general
    \item overview of the old completion (pre-AST)
    \item overview of the AST alphabetical completion and the role the sorter plays
\end{itemize}
\chapter{N-gram Background}
\label{chap:NgramBackground}

In this chapter, we go over the theoretical background for the N-grams models and the challenges in their implementation.

\section{N-gram Language Models}
\label{sec:NgramBackground-LanguageModels}
Predicting the next word in a sequence is a central problem for many areas of Natural Language Processing (NLP), including speech recognition, spelling correction, spam filtering, machine translation, and so on.

Models that assign probabilities to sentences or sequences of words, and can be used to find the most likely continuation of a sequence, are called language models (\cite{Jura09a}). Among them, is the n-gram language model, which assigns probabilities to sequences out of \textit{n} words, called the n-grams. A one word sequence is a unigram, pairs of words are referred to as 2-grams or bigrams, 3-grams or trigrams are sequences of three words, and so on. Examples of bigrams might be "he ate", "ate the", "the whole" and "whole pizza", whereas trigrams would look like "he ate the", "ate the whole", and "the whole pizza". 

N-gram language models estimate the likelihood of a word in a sequence, based on the "history", which is the previous \textit{(n-1)} words.

\section{N-grams}
\label{sec:NgramBackground-Ngrams}
To describe the conditional probability of a word \textit{w} based on history \textit{h}, we use the following notation:
\begin{equation}
    P(w|h)
\end{equation}

Here is a more concrete example: if we have a sentence "he ate the whole pizza", and want to compute the probability of the word "pizza" given that the previous words are "he ate the whole", we can express it as a conditional probability:
\begin{equation}
    P(\text{pizza}|\text{he ate the whole})
\end{equation}

To estimate it, we need to compute the number of occurrences of "he ate the whole", as well the number occurrences of the sentence "he ate the whole pizza", and divide the latter by the former:
\begin{equation}
    P(\text{pizza}|\text{he ate the whole})=\frac{C(\text{he ate the whole pizza})}{C(\text{he ate the whole})}
\end{equation}

It is worth noting that history \textit{h} can include all previous words in a text and \textit{n} can be quite large, too, so in a sizeable data corpora it can be a challenge to estimate such big sequences. However, the idea behind the n-gram model is that we do not necessarily need a large history and can restrict it to just the \textit{n-1} previous words. This is done by using the Markov property, which states that the probability of future states depends only on the present state, and not on the sequence of events that preceded it. Thus, we can generalise the bigram (which looks one word into the past) to the trigram, and thus to the n-gram (\cite{Jura09a}). This means we can reduce our conditional probabilities as follows:
\begin{equation}
    P(\text{pizza}|\text{he ate the whole}) \sim P(\text{pizza}|\text{whole})
\end{equation}

One of the most straightforward ways to estimate the probability of n-grams is using maximum likelihood estimation, or MLE. For instance, to compute a probability of a word \textit{w} given previous word \textit{h}, one can determine the count of the bigram C(hw) and normalize it by the sum of all the bigrams that also start with \textit{h}. An example of MLE is estimating relative frequencies of a corpus. The idea is that we estimate the n-gram probability by dividing the number of occurrences (frequency) of a sequence by the total number of sequences in the dataset. \notclear{need help with adding MLE and relative frequency explanations?} \mr{HELP !!!}

\section{Challenges}
\label{sec:NgramBackground-Challenges}
A notable problem with the MLE approach is sparse data in a corpus. There might be not enough observations to estimate probabilities well simply by counting observed data, because depending on the existing vocabulary, common words and words that are barely used can both have a count of 1. Or even words that are likely to appear in real life, can somehow be absent from the vocabulary, and the probability of encountering them will therefore be calculated as 0.

This, in turn, can be quite a problem even for the whole sequence: seeing as the joint probability of a sequence is calculated as a product of all conditional probabilities, even if one multiplier is 0, the the product (i.e. the probability of the sequence) will also be 0. This can significantly hurt the performance of a model.

To solve this problem, smoothing is used.

\section{Smoothing}
\label{sec:NgramBackground-Smoothing}
If a word is unknown, we can replace it with the unknown word token <UNK> and count it as any other word in the dataset. However, if the word we are dealing with is generally known (does exist in the vocabulary) but appears in an unknown context, i.e. after a word they previously never appeared next to, we can use smoothing, which means we redistribute the general probability to not have any sequences with zero probability. \oz{This explanation is not clear} \mr{did i fix it?}

There are several smoothing algorithms, the simplest of which is Laplace smoothing. In this algorithm, we add 1 to all the counts (so all the counts will be increased by one, and there will be no 0 counts), and then add $|V|$ (the size of a vocabulary = number of unique words in it) to the denominator, to take into account the extra observations. So the smoothed MLE of the unigram probability of a word \textit{w} will look the following:
\begin{equation}
    P(w)=\frac{c+1}{N+V}
\end{equation}
where \textit{c} is the count of the word \textit{w} and \textit{N} is the total number of words.

\section{Summary}
\label{sec:NgramBackground-Summary}
\begin{itemize}
    \item Language models assign probabilities to sentences and sequences of words, and can be used to predict next word in a sequence.
    \item N-gram language models calculate conditional probabilities of words in a corpus given a limited history of \textit{n-1} previous words.
    \item N-gram models belong to the family of Markov models -- they make a restrictive assumption that every word in a sequence depends only on \textit{n-1} previous words, and any words beyond that window have little effect on probability and can be ignored.
    \item Smoothing is needed in case words appear in an unknown context or are simply not in the vocabulary; that way probability is redistributed among all the sequences to prevent skewing of data, i.e. when the joint probability of a whole sequence is zero only because the count (and probability) for one single word in the sequence was also zero.
\end{itemize}
\chapter{Proposed Solution}
\label{chap:ProposedSolution}

\section{Background}
\label{sec:ProposedSolution-Background}
In our case, in code completion, we want to suggest the most likely tokens at every step. The main idea is, however, that we leave the existing, AST-based code completion engine in place, and just enhance the sorting strategy. That is, after we already have a list of completion candidates proposed to us, we sort it based on each ngram's probability and therefore show the most relevant completions first. In order to do that, we figured it would make sense to train our model on source code data, write it to file, and at every point fetch the relevant, pre-processed information from file where it is stored.

The source code history data used for this was retrieved by \cite{Zait20a} for their research regarding characterising Pharo code. Specifically, the data comes from 50 projects, which consisted of 824 packages, 13,935 classes, and 151,717 methods. In particular, we used the dataset where the source code was already split into tokens and respective token types for each method. In this dataset, even delimiters and non-alphabetic tokens were included, and comments were considered tokens. For example, tokens could include '"Here is a comment for this method"' and '.', and their respective token types would be 'COM' and 'DOT'.

Due to a large number of tokens in our dataset, we needed to be mindful of potential time constraints, as the lookup of ngram probabilities used for sorting had to be fast enough to not make the developer pause and wait for it. Throughout the course of the experiment, we came up with various ways to additionally reduce the dataset, and each of those attempts will be described in the following sections.

\section{Solution Overview}
\label{sec:ProposedSolution-Overview}
We implemented two separate sorting strategies based on the N-gram approach.

The first one, unigram sorting, is based on 1-gram analysis, which means that only the token we are trying to complete is being taken into consideration. The implementation of the unigram model itself comes down to calculating individual token frequencies, i.e. the number of occurrences of each token in the source code history. Then the sorting strategy is implemented by utilising the frequency information and sorting the completion candidates we are given based on each one's number of occurences.

The second, more advanced one, is the bigram sorting strategy. Being an N-gram model with N=2, it relies on both the token currently being completed, as well as the previous history -- in this case, one token before the current one. To make it work, we learnt how to infer the previous token from the context of the method in which we are typing, as well as calculated the probability of each of the ngram sequences. By which we mean that we assembled the history word and each completion candidate into respective pair of tokens and calculated the relative frequency of each one appearing in the source code history, and then based on that returned the probability. The sorting of results per the bigram sorting strategy is thus based on this probability of the generated bigram sequences but is done only for the completion candidates (i.e. words we are predicting in the bigram model), as they are the ones we are interested in seeing displayed in the most relevant order.

The unigram-based sorter is available in the Pharo IDE as the FrequencyCompletionSorter, and the bigram-based as the BigramCompltionSorter -- i.e. they can already be used by enabling the respective sorting option in the Settings. \mr{add screenshot}

\section{Implementation Details}
\label{sec:ProposedSolution-Implementation}
Before training our models, we took some data preparation steps, additionally cleaning the data, such as deleting some rows with token and token type mismatch, eliminating double tabs and replacing them with placeholders, and splitting token and token types into separate columns. After that, when recording token frequencies for the unigram model, we decided we would write the results to file, for faster lookup for the sorting. However, even with that approach, we faced a problem: as the initial number of tokens was around 150k, the lookup was slow enough to be noticeable and break the developer's flow when typing. 

Our approach to improve the time efficiency was to set a certain threshold for the number of occurences, and to cut off all the tokens with frequencies below it. We put the threshold equal to 10, meaning if the token occured less than 10 times throughout the whole history, it was irrelevant enough to be discarded. This way, we were able to cut off most of the miscellaneous, rarely encountered tokens, and shortened our dataset from 150k to just 16k entries. This made the frequencies lookup during completion sorting very fast and, as a result, it was now possible to type and use completion information without any delays.

The one-time tokens, such as custom strings and comments, were not as relevant to us during implementing the unigram sorting strategy, however when we moved on to implementing the bigram sorter, the bigram model became "too heavy" due to many combinations of such tokens, and needed to be additionally reduced. Our solution was to go back to the data processing step and replace those "one-time" tokens with placeholders. In particular, we added placeholders for strings, comments, symbols, characters, arrays and numbers (such as all strings being written as <str> and all numbers as <num>). This allowed us to greatly reduce both the number of total tokens and subsequent ngram sequences, which helped both the bigram and the unigram model, as well as the lookup speed.

\section{Implementation Details: Library Extension}
\label{sec:ProposedSolution-LibraryExtension}
The bigram model trained on source code data was implemented with the help of the NgramModel library \footnote{\url{https://github.com/pharo-ai/NgramModel}}. In the process of working on implementing the model, we extended the existing library with the following functionality:
\begin{itemize}
    \item functionality to filter the table of ngrams (cut off ngram sequences with counts below a certain threshold)
    \item adding writing to and reading ngram models from file
\end{itemize}

\section{Summary}
\label{sec:ProposedSolution-Summary}
A short summary of the implementation, along with data preparation and engineering details:
\begin{itemize}
    \item the data used to train the models came from 50 projects implemented in Pharo; in the dataset the source code was already split into tokens and token types
    \item in the data, delimiters and non-alphabetic tokens, as well as comments, were all considered to be separate tokens
    \item for unigram sorting, individual token frequencies were extracted; the sorting of completion candidates was based on the number of occurences of each token by itself
    \item for bigram sorting, completion candidates were added to the previous token to form sequences of bigrams, whose probability was later calculated and recorded to file; the sorting was based on the sequence probability but it was only applied to individually displayed completion candidates
    \item both sorting strategies can be enabled to be used in the Pharo IDE
    \item different approaches were taken to reduce the sizes of models and speed up lookup and sorting time; among them: cleaning up tokens and sequences with occurences below a certain threshold, and replacing irrelevant "one-time" tokens (such as specific strings, numbers, characters, and so on) with general placeholders (such as <str>, <num>, etc.)
    \item in the process of working on the bigram sorting strategy, the NgramModel Pharo library that was used for this approach was addiitionally extended with new functionality
\end{itemize}
\chapter{Evaluation}
\label{chap:Evaluation}

\section{Evaluation Overview}
overview of evaluation strategies and why it's important to evaluate code

\section{Challenges in Evaluating Completion}
describing why it's not so easy to evaluate code completion + the evaluation approach we've decided to take

\section{The Experiment Itself}
describing what we did and how we set it up exactly

\section{Results}
demonstrate what we got as the result of evaluating and comparing different sorting strategies
% \chapter{Conclusion}
\label{chap:Conclusion}
\section{Discoveries}
\label{sec:Conclusion-Discoveries}
In this work, I proposed a machine learning-based technique for improving the completion engine in the Pharo IDE. The exact implementation used the n-gram language models to sort the completion candidates that are suggested to the user \replace{during coding}{as he or she types}. \remove{Specifically,} \insertion{I implemented} two sorting strategies \remove{were implemented}: \remove{the first} one \remove{was} based on the unigram model\remove{,} and \replace{the second}{another} one on the bigram model. \oz{The rest of this paragraph hardly fits into the Conclusion chapter. But it would be really nice to say this in the chapter where you describe your proposed solution} N-gram models of a higher order, such as 3- and 4-gram models, were not considered due \replace{the difficulty of implementation}{their computational complexity}: the \replace{size of the data}{frequency table} increases \replace{dramatically}{exponentially} with every new order of \textit{n}\replace{, and making it fast enough to be usable would have been a really time-consuming task beyond the scope of this project}{which makes the calculation of probabilities too slow for the task, where results must be updated at every keystroke}.

Concluding this work, the answers to the research questions originally stated in Section \ref{sec:Introduction-Approach} go as follows:
\begin{RQ}
    \item \textbf{Can we improve the accuracy of code completion in the Pharo IDE by sorting candidate completions with n-gram language models?} As a result of the evaluation performed, the unigram based sorter has been shown to have a significantly better result than the default sorter in the Pharo IDE. The implementation is also fast enough for comfortable developer usage, and is available by loading the \href{https://github.com/myroslavarm/CompletionSorting}{CompletionSorting} library. \oz{Never hide URL behind the link text. What if this document is printed on a paper? You can write "CompletionSorting library available at https://..."}
    \item \textbf{How can we effectively evaluate the results of code completion enhanced by different sorting strategies?} Even \wrong{with an evident resources limitation} \oz{???!!!}, it \wrong{has been possible to come up with an approach which adequately measures the performance of the sorters}. \oz{how do you define "adequately"? For example, I don't agree with you: I doubt that evaluation was adequate and think that it can be improved. So don't write this sentence} In particular, for the quantitative evaluation, inspired by \cite{Robb08a}, I \remove{was able to} simulate\insertion{d} the completion process as it would happen naturally, by generating source code sequence at various stages of typing, and comparing the results proposed to the ones in the codebase. The qualitative approach \replace{is of interest from the standpoint of being able}{allowed me} to compare the suggestions each sorting strategy proposes first-hand \oz{and...? This sentence sounds like it's not finished}.
\end{RQ}

\section{Directions of Future Work}
\label{sec:Conclusion-FutureWork}
\subsection{Enhancing the Bigram Sorter}

\subsection{Conducting a More Extensive Evaluation}


%----------------------------------------------------------------------------------------
%	THESIS CONTENT - APPENDICES
%----------------------------------------------------------------------------------------

\appendix % Cue to tell LaTeX that the following "chapters" are Appendices

% Include the appendices of the thesis as separate files from the Appendices folder
% Uncomment the lines as you write the Appendices

%\include{Appendices/AppendixA}
%\include{Appendices/AppendixB}
%\include{Appendices/AppendixC}

%----------------------------------------------------------------------------------------
%	BIBLIOGRAPHY
%----------------------------------------------------------------------------------------

\printbibliography[heading=bibintoc]

%----------------------------------------------------------------------------------------

\end{document}  
